\documentclass[12pt,a4paper]{article}
\usepackage[utf8]{inputenc}
\usepackage[T1]{fontenc}
\usepackage{geometry}
\usepackage{graphicx}
\usepackage{hyperref}
\usepackage{listings}
\usepackage{xcolor}
\usepackage{booktabs}
\usepackage{multirow}
\usepackage{enumitem}
\usepackage{fancyhdr}
\usepackage{titlesec}
\usepackage{float}
\usepackage{caption}
\usepackage{subcaption}

% Page setup
\geometry{margin=2.5cm}
\pagestyle{fancy}
\fancyhf{}
\fancyhead[L]{CSC2209 - Database Programming}
\fancyhead[R]{Uganda Christian University}
\fancyfoot[C]{\thepage}

% Title formatting
\titleformat{\section}{\Large\bfseries}{\thesection}{1em}{}
\titleformat{\subsection}{\large\bfseries}{\thesubsection}{1em}{}
\titleformat{\subsubsection}{\normalsize\bfseries}{\thesubsubsection}{1em}{}

% Code listing setup
\lstset{
    language=SQL,
    basicstyle=\ttfamily\small,
    keywordstyle=\color{blue}\bfseries,
    commentstyle=\color{green!60!black},
    stringstyle=\color{red},
    numbers=left,
    numberstyle=\tiny\color{gray},
    stepnumber=1,
    numbersep=5pt,
    backgroundcolor=\color{gray!10},
    frame=single,
    breaklines=true,
    breakatwhitespace=true,
    tabsize=2,
    showspaces=false,
    showstringspaces=false
}

% Document metadata
\title{
    \vspace{-2cm}
    \textbf{DESIGN AND IMPLEMENTATION OF A DATABASE APPLICATION}\\
    \large HIV Patient Care \& Treatment Monitoring System\\
    \large Mukono General Hospital ART Clinic
    \vspace{0.5cm}
}
\author{
    \textbf{Student Name:} [Your Name]\\
    \textbf{Registration Number:} [Your Registration Number]\\
    \textbf{Program:} Bachelor of Science in Data Science and Analytics\\
    \textbf{Course:} CSC2209 - Database Programming\\
    \textbf{Institution:} Uganda Christian University\\
    \textbf{Academic Year:} 2024/2025\\
    \textbf{Semester:} Advent 2025\\
    \vspace{0.3cm}
    \textbf{Project Repository:} \url{https://github.com/Cemputus/Database-Programming-Research-Project}
}
\date{\today}

\begin{document}

\maketitle

\thispagestyle{empty}

\newpage

\tableofcontents
\newpage

\listoffigures
\newpage

% ============================================================================
% EXECUTIVE SUMMARY
% ============================================================================
\section{Executive Summary}

This project presents the design and implementation of a comprehensive database application for managing HIV patient care and treatment monitoring at Mukono General Hospital ART Clinic, Uganda. The system addresses critical challenges in paper-based record keeping that impede effective patient care management for approximately 1.49 million people living with HIV in Uganda.

\textbf{Project Repository:} The complete source code and documentation are available at \url{https://github.com/Cemputus/Database-Programming-Research-Project}

The solution aligns with \textbf{SDG 3: Good Health and Well-being} by improving healthcare delivery, treatment adherence, and patient outcomes through digital transformation. The system implements an Enhanced Entity Relationship (EER) model with advanced database features including stored procedures, triggers, views, role-based access control, and automated alerts.

\textbf{Key Achievements:}
\begin{itemize}
    \item 17 core tables with comprehensive relationships
    \item 15+ views for reporting and patient self-service
    \item 21 stored procedures for business logic
    \item 8 triggers for automated alerts and audit logging
    \item 5 scheduled events for routine maintenance
    \item 7 role-based access control levels
    \item Complete RESTful API with JWT authentication
\end{itemize}

% ============================================================================
% MILESTONE ONE: REQUIREMENTS SPECIFICATION
% ============================================================================
\section{Milestone One: Requirements Specification}

\subsection{Problem Statement}

Uganda faces a significant HIV/AIDS challenge with approximately \textbf{1.49 million people living with HIV} (5.1\% adult prevalence rate). Many health facilities, especially at lower levels, rely on paper-based record keeping, leading to:

\begin{itemize}
    \item Fragmented patient histories and lost records
    \item Lack of automated alerts for missed appointments and overdue tests
    \item Data silos preventing integrated patient care
    \item Inefficient reporting and poor decision-making
    \item Treatment failures and poor viral suppression rates
\end{itemize}

\subsection{SDG Alignment}

This project aligns with \textbf{SDG 3: Good Health and Well-being}, specifically:
\begin{itemize}
    \item Target 3.3: End epidemics of AIDS, tuberculosis, and malaria
    \item Target 3.8: Achieve universal health coverage
    \item Supporting Uganda's national HIV response goals
\end{itemize}

\subsection{Requirements Elicitation Methods}

The following methods were used to gather requirements:

\begin{enumerate}
    \item \textbf{Literature Review}: Analysis of Uganda MOH guidelines, HMIS forms, and HIV treatment protocols
    \item \textbf{Stakeholder Analysis}: Identification of key users (clinicians, lab staff, pharmacists, counselors, administrators)
    \item \textbf{Document Analysis}: Review of existing paper forms (HMIS 031, pharmacy logs, lab registers)
    \item \textbf{Use Case Modeling}: Identification of primary workflows and user interactions
\end{enumerate}

\subsection{Functional Requirements}

\subsubsection{FR1: Patient Management}
\begin{itemize}
    \item FR1.1: Register new patients with Uganda NIN and demographic information
    \item FR1.2: Track patient enrollment dates and ART initiation
    \item FR1.3: Manage patient status (Active, Transferred-Out, LTFU, Dead)
    \item FR1.4: Maintain patient location hierarchy (District → Subcounty → Parish → Village)
\end{itemize}

\subsubsection{FR2: Clinical Operations}
\begin{itemize}
    \item FR2.1: Record clinical visits with vital signs, symptoms, and diagnoses
    \item FR2.2: Track WHO clinical staging (1-4)
    \item FR2.3: Manage TB screening results
    \item FR2.4: Schedule and track patient appointments
\end{itemize}

\subsubsection{FR3: Laboratory Management}
\begin{itemize}
    \item FR3.1: Record laboratory test results (Viral Load, CD4, HB, Creatinine, Malaria RDT, TB-LAM, Urinalysis)
    \item FR3.2: Track CPHL sample IDs for viral load tests
    \item FR3.3: Manage test status (Pending, Completed, Rejected)
    \item FR3.4: Link lab tests to clinical visits
\end{itemize}

\subsubsection{FR4: Pharmacy Management}
\begin{itemize}
    \item FR4.1: Maintain ART regimen catalog (First Line, Second Line, Third Line)
    \item FR4.2: Record medication dispensing with days supply
    \item FR4.3: Calculate next refill dates automatically
    \item FR4.4: Track medication refills and identify overdue refills
\end{itemize}

\subsubsection{FR5: Counseling and Adherence}
\begin{itemize}
    \item FR5.1: Record counseling sessions with topics and barriers
    \item FR5.2: Track adherence assessments using multiple methods (Pill Count, Pharmacy Refill, Self Report, CAG Report, Computed)
    \item FR5.3: Calculate adherence percentages (0-100\%)
    \item FR5.4: Identify patients requiring adherence support
\end{itemize}

\subsubsection{FR6: Community ART Groups (CAGs)}
\begin{itemize}
    \item FR6.1: Create and manage CAGs with location information
    \item FR6.2: Assign patients to CAGs with roles (Member, Coordinator, Deputy Coordinator)
    \item FR6.3: Track CAG medication pickup rotations
    \item FR6.4: Monitor CAG performance metrics
\end{itemize}

\subsubsection{FR7: Automated Alerts}
\begin{itemize}
    \item FR7.1: Generate alerts for high viral loads (>1000 copies/mL)
    \item FR7.2: Alert for overdue viral load tests (>180 days)
    \item FR7.3: Alert for missed appointments
    \item FR7.4: Alert for missed medication refills
    \item FR7.5: Alert for low adherence (<85\%)
\end{itemize}

\subsubsection{FR8: Security and Access Control}
\begin{itemize}
    \item FR8.1: Implement role-based access control (7 roles)
    \item FR8.2: Authenticate users with secure credentials
    \item FR8.3: Audit all data modifications
    \item FR8.4: Restrict access based on user roles
\end{itemize}

\subsubsection{FR9: Reporting and Analytics}
\begin{itemize}
    \item FR9.1: Generate patient dashboards
    \item FR9.2: Create role-based analysis reports
    \item FR9.3: Export data for HMIS reporting
    \item FR9.4: Provide patient self-service views
\end{itemize}

\subsection{Assumptions}

\begin{enumerate}
    \item All patients have or can obtain Uganda National Identification Numbers (NIN)
    \item Staff members have appropriate professional qualifications and registrations
    \item The facility has reliable internet connectivity for API access
    \item Users are trained on system usage and data entry procedures
    \item Laboratory results are entered within 48 hours of test completion
    \item Medication dispensing follows Uganda MOH standard regimens
    \item CAGs operate according to Uganda MOH CAG guidelines
    \item Viral load tests are conducted every 6 months for stable patients
    \item The system will be used in a controlled environment with backup power
    \item Data migration from paper records will be done gradually
\end{enumerate}

% ============================================================================
% MILESTONE TWO: DESIGN
% ============================================================================
\section{Milestone Two: Design}

\subsection{Enhanced Entity Relationship (EER) Model}

The database design follows an Enhanced Entity Relationship model with the following features:

\subsubsection{Generalization (Supertype/Subtype)}

\textbf{Supertype:} \texttt{person}
\begin{itemize}
    \item Contains common attributes: NIN, name, date of birth, gender, location, contact
    \item Inherited by: \texttt{patient} and \texttt{staff} (subtypes)
    \item Eliminates data duplication and ensures consistency
\end{itemize}

\textbf{Subtypes:}
\begin{itemize}
    \item \texttt{patient}: Adds patient-specific attributes (patient\_code, enrollment\_date, art\_start\_date, current\_status)
    \item \texttt{staff}: Adds staff-specific attributes (staff\_code, cadre, moH\_registration\_no, hire\_date)
\end{itemize}

\subsubsection{Overlapping Specialization}

\textbf{staff\_role} junction table enables:
\begin{itemize}
    \item One staff member to have multiple roles simultaneously
    \item Example: A nurse can also be a midwife
    \item Roles represent system privileges, not job cadres
\end{itemize}

\subsubsection{Disjoint Categorization}

\textbf{patient.current\_status}:
\begin{itemize}
    \item Must be exactly one of: Active, Transferred-Out, LTFU, Dead
    \item Mutually exclusive values (disjoint)
    \item Enforced through ENUM constraint
\end{itemize}

\subsubsection{Categorization}

Multiple categorization examples:
\begin{itemize}
    \item \texttt{lab\_test.test\_type}: Viral Load, CD4, HB, Creatinine, Malaria RDT, TB-LAM, Urinalysis
    \item \texttt{alert.alert\_type}: High Viral Load, Overdue VL, Missed Appointment, Missed Refill, Low Adherence, Severe OI
    \item \texttt{adherence\_log.method\_used}: Pill Count, Pharmacy Refill, Self Report, CAG Report, Computed
\end{itemize}

\subsection{Referential Integrity}

All foreign key relationships enforce referential integrity:

\begin{itemize}
    \item \texttt{patient.person\_id} → \texttt{person.person\_id} (ON DELETE RESTRICT, ON UPDATE CASCADE)
    \item \texttt{visit.patient\_id} → \texttt{patient.patient\_id} (ON DELETE RESTRICT, ON UPDATE CASCADE)
    \item \texttt{lab\_test.patient\_id} → \texttt{patient.patient\_id} (ON DELETE RESTRICT, ON UPDATE CASCADE)
    \item \texttt{dispense.regimen\_id} → \texttt{regimen.regimen\_id} (ON DELETE RESTRICT, ON UPDATE CASCADE)
    \item \texttt{patient\_cag.patient\_id} → \texttt{patient.patient\_id} (ON DELETE RESTRICT, ON UPDATE CASCADE)
    \item \texttt{patient\_cag.cag\_id} → \texttt{cag.cag\_id} (ON DELETE RESTRICT, ON UPDATE CASCADE)
\end{itemize}

\subsection{Business Rules}

Key business rules enforced through constraints and triggers:

\begin{enumerate}
    \item \textbf{ART Start Date Rule}: ART start date must be on or after enrollment date
    \item \textbf{WHO Stage Range}: WHO clinical stage must be between 1 and 4
    \item \textbf{Days Supply Range}: Medication days supply must be between 1 and 365
    \item \textbf{Adherence Range}: Adherence percentage must be between 0 and 100\%
    \item \textbf{Exit Date Rule}: Patient CAG exit date must be on or after join date
    \item \textbf{Viral Load Non-Negative}: Viral load results cannot be negative
    \item \textbf{Unique Constraints}: NIN, patient codes, staff codes, CAG names must be unique
\end{enumerate}

\subsection{Entity Relationship Diagram}

The EER diagram (see Figure \ref{fig:eer}) illustrates:
\begin{itemize}
    \item 17 core entities with relationships
    \item Generalization hierarchy (person → patient/staff)
    \item Many-to-many relationships (patient ↔ CAG, staff ↔ role)
    \item One-to-many relationships (patient → visits, lab tests, dispenses)
\end{itemize}

\begin{figure}[H]
    \centering
    \includegraphics[width=0.95\textwidth]{hiv_patient_care.png}
    \caption{Enhanced Entity Relationship Diagram - HIV Patient Care \& Treatment Monitoring System}
    \label{fig:eer}
\end{figure}

% ============================================================================
% MILESTONE THREE: DATABASE DEVELOPMENT
% ============================================================================
\section{Milestone Three: Database Development}

\subsection{Database Structure}

The database consists of 17 core tables organized as follows:

\subsubsection{Core Tables}

\begin{description}
    \item[\texttt{person}] Supertype table with 11 attributes including NIN, names, demographics, location
    \item[\texttt{patient}] Subtype with 9 attributes including patient\_code, enrollment\_date, art\_start\_date, current\_status
    \item[\texttt{staff}] Subtype with 6 attributes including staff\_code, cadre, moH\_registration\_no
    \item[\texttt{role}] System access roles (7 roles: db\_admin, db\_clinician, db\_lab, db\_pharmacy, db\_counselor, db\_readonly, db\_patient)
    \item[\texttt{staff\_role}] Junction table for many-to-many relationship
    \item[\texttt{visit}] Clinical visit records with 9 attributes
    \item[\texttt{lab\_test}] Laboratory test results with 10 attributes
    \item[\texttt{regimen}] ART regimen catalog with 4 attributes
    \item[\texttt{dispense}] Medication dispensing records with 8 attributes
    \item[\texttt{appointment}] Appointment scheduling with 5 attributes
    \item[\texttt{counseling\_session}] Counseling session records with 5 attributes
    \item[\texttt{cag}] Community ART Groups with 11 attributes
    \item[\texttt{patient\_cag}] Junction table for patient-CAG membership
    \item[\texttt{cag\_rotation}] CAG medication pickup rotations with 7 attributes
    \item[\texttt{adherence\_log}] Adherence assessment records with 5 attributes
    \item[\texttt{alert}] Automated alerts with 8 attributes
    \item[\texttt{audit\_log}] Audit trail with 6 attributes
\end{description}

\subsection{Data Validation Constraints}

\subsubsection{Primary Key Constraints}

All tables have primary keys ensuring unique identification:
\begin{lstlisting}
PRIMARY KEY (`patient_id`)
PRIMARY KEY (`visit_id`)
PRIMARY KEY (`lab_test_id`)
\end{lstlisting}

\subsubsection{Foreign Key Constraints}

All relationships enforce referential integrity:
\begin{lstlisting}
CONSTRAINT `fk_patient_person` 
    FOREIGN KEY (`person_id`) 
    REFERENCES `person` (`person_id`) 
    ON DELETE RESTRICT ON UPDATE CASCADE
\end{lstlisting}

\subsubsection{Unique Constraints}

Critical attributes have unique constraints:
\begin{lstlisting}
UNIQUE KEY `uk_nin` (`nin`)
UNIQUE KEY `uk_patient_code` (`patient_code`)
UNIQUE KEY `uk_staff_code` (`staff_code`)
UNIQUE KEY `uk_cag_name` (`cag_name`)
\end{lstlisting}

\subsubsection{Check Constraints}

Business rules enforced through CHECK constraints:
\begin{lstlisting}
CONSTRAINT `chk_art_after_enrollment` 
    CHECK (`art_start_date` IS NULL OR 
           `art_start_date` >= `enrollment_date`)

CONSTRAINT `chk_who_stage_range` 
    CHECK (`who_stage` IS NULL OR 
           (`who_stage` >= 1 AND `who_stage` <= 4))

CONSTRAINT `chk_days_supply_range` 
    CHECK (`days_supply` > 0 AND `days_supply` <= 365)

CONSTRAINT `chk_adherence_range` 
    CHECK (`adherence_percent` >= 0 AND 
           `adherence_percent` <= 100)

CONSTRAINT `chk_exit_date_after_join` 
    CHECK (`exit_date` IS NULL OR 
           `exit_date` >= `join_date`)
\end{lstlisting}

\subsubsection{ENUM Constraints}

Categorical data restricted to valid values:
\begin{lstlisting}
`sex` ENUM('M', 'F')
`current_status` ENUM('Active', 'Transferred-Out', 'LTFU', 'Dead')
`test_type` ENUM('Viral Load', 'CD4', 'HB', 'Creatinine', 
                 'Malaria RDT', 'TB-LAM', 'Urinalysis')
`result_status` ENUM('Pending', 'Completed', 'Rejected')
`alert_type` ENUM('High Viral Load', 'Overdue VL', 
                   'Missed Appointment', 'Missed Refill', 
                   'Low Adherence', 'Severe OI')
\end{lstlisting}

\subsection{Indexes for Performance}

Strategic indexes created for query optimization:

\begin{lstlisting}
CREATE INDEX `idx_patient_status_enrollment` 
    ON `patient` (`current_status`, `enrollment_date`);

CREATE INDEX `idx_visit_patient_date` 
    ON `visit` (`patient_id`, `visit_date` DESC);

CREATE INDEX `idx_lab_test_patient_type_date` 
    ON `lab_test` (`patient_id`, `test_type`, `test_date` DESC);

CREATE INDEX `idx_dispense_patient_date` 
    ON `dispense` (`patient_id`, `dispense_date` DESC);
\end{lstlisting}

\subsection{Data Validation Functions}

Built-in MySQL functions used for validation:
\begin{itemize}
    \item \texttt{CURDATE()} - Current date validation
    \item \texttt{DATEDIFF()} - Date difference calculations
    \item \texttt{TIMESTAMPDIFF()} - Age calculations
    \item \texttt{COALESCE()} - Null value handling
    \item \texttt{GREATEST()} - Maximum value selection
\end{itemize}

% ============================================================================
% MILESTONE FOUR: SECURITY AND AUTOMATION
% ============================================================================
\section{Milestone Four: Security and Automation}

\subsection{Views}

\subsubsection{Clinical Views}

\begin{itemize}
    \item \texttt{v\_active\_patients\_summary} - Active patients with key metrics
    \item \texttt{v\_patient\_care\_timeline} - Chronological patient events
    \item \texttt{v\_viral\_load\_monitoring} - VL status and test scheduling
    \item \texttt{v\_adherence\_summary} - Latest adherence assessments
    \item \texttt{v\_active\_alerts\_summary} - All unresolved alerts
\end{itemize}

\subsubsection{Patient Self-Service Views}

\begin{itemize}
    \item \texttt{v\_patient\_dashboard} - Complete patient overview
    \item \texttt{v\_patient\_visit\_history} - Clinical visit history
    \item \texttt{v\_patient\_lab\_history} - Lab test results
    \item \texttt{v\_patient\_medication\_history} - Medication records
    \item \texttt{v\_patient\_appointments} - Appointment schedule
    \item \texttt{v\_patient\_adherence\_history} - Adherence assessments
    \item \texttt{v\_patient\_alerts} - Patient alerts
    \item \texttt{v\_patient\_progress\_timeline} - Complete care timeline
\end{itemize}

\subsubsection{CAG Views}

\begin{itemize}
    \item \texttt{v\_cag\_summary} - CAG overview with member counts
    \item \texttt{v\_cag\_members} - Active CAG members
    \item \texttt{v\_cag\_rotation\_history} - Rotation records
    \item \texttt{v\_cag\_performance} - CAG performance metrics
\end{itemize}

\subsection{User Authentication, Privileges and Roles}

The system implements comprehensive role-based access control with seven distinct roles. Figure \ref{fig:admin} demonstrates the role-based analysis interface showing staff distribution by cadre.

\begin{figure}[H]
    \centering
    \includegraphics[width=0.9\textwidth]{admin.png}
    \caption{Role-Based Access Control - Administrator View: Staff Analysis by Cadre}
    \label{fig:admin}
\end{figure}

\subsubsection{Database Roles}

Seven distinct roles created:

\begin{enumerate}
    \item \textbf{db\_admin} - Full system access (ALL PRIVILEGES)
    \item \textbf{db\_clinician} - Read all, write to visits, appointments, patients, lab\_tests, adherence\_log, alerts
    \item \textbf{db\_lab} - Read patient/staff data, write lab\_test results
    \item \textbf{db\_pharmacy} - Read patient/regimen data, write dispense records
    \item \textbf{db\_counselor} - Read patient data, write counseling\_session, adherence\_log, appointments
    \item \textbf{db\_readonly} - Read-only access to all tables and views
    \item \textbf{db\_patient} - Read-only access to own data via views and procedures
\end{enumerate}

\subsubsection{Role Implementation}

\begin{lstlisting}
-- Create roles
CREATE ROLE 'db_admin';
CREATE ROLE 'db_clinician';
CREATE ROLE 'db_lab';
CREATE ROLE 'db_pharmacy';
CREATE ROLE 'db_counselor';
CREATE ROLE 'db_readonly';
CREATE ROLE 'db_patient';

-- Grant privileges
GRANT ALL PRIVILEGES ON hiv_patient_care.* TO 'db_admin';
GRANT SELECT ON hiv_patient_care.* TO 'db_clinician';
GRANT INSERT, UPDATE ON hiv_patient_care.`visit` TO 'db_clinician';

-- Create users and assign roles
CREATE USER 'doctor1'@'localhost' IDENTIFIED BY 'doctor123';
GRANT 'db_clinician' TO 'doctor1'@'localhost';
SET DEFAULT ROLE 'db_clinician' TO 'doctor1'@'localhost';
\end{lstlisting}

\subsection{Stored Procedures}

\subsubsection{Patient Self-Service Procedures}

\begin{itemize}
    \item \texttt{sp\_patient\_dashboard} - Get complete patient dashboard
    \item \texttt{sp\_patient\_visits} - Get visit history
    \item \texttt{sp\_patient\_lab\_tests} - Get lab test history
    \item \texttt{sp\_patient\_medications} - Get medication history
    \item \texttt{sp\_patient\_appointments} - Get appointments
    \item \texttt{sp\_patient\_adherence} - Get adherence history
    \item \texttt{sp\_patient\_alerts} - Get patient alerts
    \item \texttt{sp\_patient\_progress\_timeline} - Get care timeline
\end{itemize}

\subsubsection{Clinical Operations Procedures}

\begin{itemize}
    \item \texttt{sp\_compute\_adherence} - Calculate adherence percentage
    \item \texttt{sp\_check\_overdue\_vl} - Check for overdue VL tests
    \item \texttt{sp\_mark\_missed\_appointments} - Mark missed appointments
    \item \texttt{sp\_update\_patient\_status\_ltfu} - Update LTFU status
    \item \texttt{sp\_check\_missed\_refills} - Check for missed refills
\end{itemize}

\subsubsection{CAG Management Procedures}

\begin{itemize}
    \item \texttt{sp\_cag\_add\_patient} - Add patient to CAG
    \item \texttt{sp\_cag\_remove\_patient} - Remove patient from CAG
    \item \texttt{sp\_cag\_record\_rotation} - Record CAG rotation
    \item \texttt{sp\_cag\_get\_members} - Get CAG members
    \item \texttt{sp\_cag\_get\_rotations} - Get rotation history
    \item \texttt{sp\_cag\_get\_statistics} - Get CAG statistics
\end{itemize}

\subsubsection{Example Stored Procedure}

\begin{lstlisting}
CREATE PROCEDURE `sp_compute_adherence`(
    IN p_patient_id INT UNSIGNED,
    IN p_start_date DATE,
    IN p_end_date DATE
)
BEGIN
    DECLARE v_pills_expected INT UNSIGNED;
    DECLARE v_pills_missed INT UNSIGNED;
    DECLARE v_adherence_percentage DECIMAL(5,2);
    
    -- Calculate adherence from dispense records
    SELECT SUM(quantity_dispensed) INTO v_pills_expected
    FROM dispense
    WHERE patient_id = p_patient_id
      AND dispense_date BETWEEN p_start_date AND p_end_date;
    
    -- Calculate adherence percentage
    SET v_adherence_percentage = 
        ((v_pills_expected - v_pills_missed) / v_pills_expected) * 100;
    
    -- Save to adherence log
    INSERT INTO adherence_log (
        patient_id, log_date, method_used, 
        adherence_percent, notes
    ) VALUES (
        p_patient_id, CURDATE(), 'Computed',
        v_adherence_percentage, 'Computed from dispense records'
    );
    
    SELECT v_adherence_percentage AS adherence_percentage;
END
\end{lstlisting}

\subsection{Triggers}

\subsubsection{Alert Triggers}

\begin{itemize}
    \item \texttt{trg\_lab\_test\_high\_vl\_alert} - Creates alert on high VL (>1000) insertion
    \item \texttt{trg\_lab\_test\_high\_vl\_alert\_update} - Creates alert on VL update
    \item \texttt{trg\_appointment\_missed\_alert} - Creates alert on missed appointment
    \item \texttt{trg\_adherence\_low\_alert} - Creates alert on low adherence (<85\%)
\end{itemize}

\subsubsection{Audit Triggers}

\begin{itemize}
    \item \texttt{trg\_patient\_audit\_insert} - Logs patient creation
    \item \texttt{trg\_patient\_audit\_update} - Logs patient updates
    \item \texttt{trg\_visit\_audit\_insert} - Logs visit creation
    \item \texttt{trg\_lab\_test\_audit\_insert} - Logs lab test creation
    \item \texttt{trg\_lab\_test\_audit\_update} - Logs lab test updates
    \item \texttt{trg\_dispense\_audit\_insert} - Logs dispense creation
\end{itemize}

\subsubsection{Example Trigger}

\begin{lstlisting}
CREATE TRIGGER `trg_lab_test_high_vl_alert`
AFTER INSERT ON `lab_test`
FOR EACH ROW
BEGIN
    IF NEW.test_type = 'Viral Load' 
       AND NEW.result_status = 'Completed' 
       AND NEW.result_numeric IS NOT NULL 
       AND NEW.result_numeric > 1000 THEN
        
        INSERT INTO alert (
            patient_id, alert_type, alert_level,
            alert_msg, is_resolved
        ) VALUES (
            NEW.patient_id, 'High Viral Load', 'Critical',
            CONCAT('High viral load detected: ', 
                   NEW.result_numeric, ' copies/mL. Action required.'),
            FALSE
        );
    END IF;
END
\end{lstlisting}

\subsection{Scheduled Events}

Five scheduled events for automation:

\begin{enumerate}
    \item \texttt{evt\_daily\_check\_overdue\_vl} - Daily at 8 AM, checks for overdue VL tests
    \item \texttt{evt\_daily\_check\_missed\_appointments} - Daily at 8 AM, marks missed appointments
    \item \texttt{evt\_daily\_check\_missed\_refills} - Daily at 8 AM, checks for missed refills
    \item \texttt{evt\_daily\_update\_ltfu} - Daily at 7 AM, updates LTFU status
    \item \texttt{evt\_weekly\_compute\_adherence} - Weekly at 9 AM, computes adherence for all patients
\end{enumerate}

\subsubsection{Example Event}

\begin{lstlisting}
CREATE EVENT `evt_daily_check_overdue_vl`
ON SCHEDULE EVERY 1 DAY
STARTS CURRENT_DATE + INTERVAL 1 DAY + INTERVAL 8 HOUR
DO
BEGIN
    CALL sp_check_overdue_vl();
END
\end{lstlisting}

\subsection{Backup and Recovery Strategies}

\subsubsection{Database Backup}

\begin{lstlisting}
-- Full database backup
mysqldump -u root -p hiv_patient_care > 
    backup_hiv_patient_care_$(date +\%Y\%m\%d).sql

-- Backup with structure and data
mysqldump -u root -p --routines --triggers 
    hiv_patient_care > full_backup.sql
\end{lstlisting}

\subsubsection{Recovery Procedures}

\begin{enumerate}
    \item \textbf{Point-in-Time Recovery}: Using binary logs for transaction recovery
    \item \textbf{Full Backup Restoration}: Restore from daily backup files
    \item \textbf{Incremental Backup}: Daily full backups with transaction log archiving
    \item \textbf{Disaster Recovery Plan}: Off-site backup storage and recovery procedures
\end{enumerate}

\subsubsection{Backup Schedule}

\begin{itemize}
    \item \textbf{Daily}: Full database backup at 2:00 AM
    \item \textbf{Weekly}: Archive backups to external storage
    \item \textbf{Monthly}: Long-term backup retention
    \item \textbf{Before Major Updates}: Pre-update backup creation
\end{itemize}

% ============================================================================
% MILESTONE FIVE: DOCUMENTATION AND DISSEMINATION
% ============================================================================
\section{Milestone Five: Documentation and Dissemination}

\subsection{System Testing}

\subsubsection{Data Integrity Testing}

The system was tested using \texttt{verify\_data.sql} which checks:
\begin{itemize}
    \item Table record counts (17 tables verified)
    \item Foreign key integrity (no orphaned records)
    \item Data consistency (date validations, range checks)
    \item Uniqueness constraints (no duplicate keys)
    \item Enum value validity (all values within allowed sets)
    \item Referential data completeness
\end{itemize}

\subsubsection{Functional Testing}

\begin{itemize}
    \item \textbf{Patient Registration}: Successfully creates patient records with NIN validation
    \item \textbf{Visit Recording}: Correctly links visits to patients and staff
    \item \textbf{Lab Test Entry}: Validates test types and result formats
    \item \textbf{Medication Dispensing}: Calculates next refill dates correctly
    \item \textbf{Alert Generation}: Triggers fire correctly for high VL, missed appointments
    \item \textbf{Role-Based Access}: Users can only access permitted data
    \item \textbf{Stored Procedures}: All 21 procedures execute successfully
\end{itemize}

\subsubsection{Performance Testing}

\begin{itemize}
    \item Query execution times measured for common operations
    \item Index effectiveness verified through EXPLAIN plans
    \item Database size: Approximately 1.7MB with sample data
    \item Response times: Sub-second for most queries
\end{itemize}

\subsection{System Deliverables}

\subsubsection{Database Files}

\begin{itemize}
    \item \texttt{schema.sql} - Database structure (17 tables)
    \item \texttt{triggers.sql} - 8 triggers
    \item \texttt{views.sql} - 15+ views
    \item \texttt{stored\_procedures.sql} - 21 stored procedures
    \item \texttt{events.sql} - 5 scheduled events
    \item \texttt{security.sql} - Roles and privileges
    \item \texttt{hiv\_patient\_care.sql} - Combined complete setup file
\end{itemize}

\subsubsection{Documentation Files}

\begin{itemize}
    \item \texttt{README.md} - Comprehensive project documentation
    \item \texttt{SETUP\_ORDER.md} - Database setup instructions
    \item \texttt{QUICK\_START.md} - Quick start guide
    \item \texttt{database\_relationships.md} - Foreign key documentation
    \item \texttt{verify\_data.sql} - Data verification script
    \item \texttt{role\_based\_analysis.sql} - Role-based analysis queries
\end{itemize}

\subsubsection{API Documentation}

\begin{itemize}
    \item RESTful API with JWT authentication
    \item 12 route modules covering all system functions
    \item Complete endpoint documentation
    \item Authentication and authorization middleware
\end{itemize}

\subsection{Key Features Demonstrated}

\begin{enumerate}
    \item \textbf{EER Modeling}: Generalization, specialization, categorization
    \item \textbf{Data Integrity}: Foreign keys, check constraints, unique constraints
    \item \textbf{Security}: Role-based access control, user authentication, audit logging
    \item \textbf{Automation}: Triggers for alerts, stored procedures for business logic
    \item \textbf{Reporting}: Views for different user roles, analytics queries
    \item \textbf{Performance}: Strategic indexing, optimized queries
    \item \textbf{Maintenance}: Scheduled events, backup strategies
\end{enumerate}

\subsection{SDG Impact}

This system directly contributes to \textbf{SDG 3: Good Health and Well-being} by:

\begin{itemize}
    \item Improving treatment adherence through automated tracking
    \item Enabling early detection of treatment failures
    \item Supporting viral suppression goals
    \item Reducing lost-to-follow-up rates
    \item Facilitating evidence-based decision making
    \item Supporting national HIV response programs
\end{itemize}

% ============================================================================
% CONCLUSION
% ============================================================================
\section{Conclusion}

This project successfully demonstrates the design and implementation of a comprehensive database application for HIV patient care management. The system addresses critical challenges in paper-based record keeping and provides a robust, secure, and automated solution for healthcare facilities in Uganda.

\textbf{Key Achievements:}
\begin{itemize}
    \item Complete EER model with all required features (generalization, specialization, categorization)
    \item Comprehensive data validation through constraints
    \item Robust security implementation with role-based access control
    \item Extensive automation through triggers and stored procedures
    \item Complete documentation and testing
    \item Alignment with SDG 3 and Uganda MOH standards
\end{itemize}

The system is ready for deployment and can significantly improve patient care outcomes, treatment adherence, and operational efficiency at healthcare facilities managing HIV patients in Uganda.

% ============================================================================
% APPENDICES
% ============================================================================
\section{Appendices}

\subsection{Appendix A: Database Schema Summary}

\textbf{Total Tables:} 17\\
\textbf{Total Views:} 15+\\
\textbf{Total Stored Procedures:} 21\\
\textbf{Total Triggers:} 8\\
\textbf{Total Scheduled Events:} 5\\
\textbf{Total Roles:} 7\\
\textbf{Total Foreign Keys:} 25+\\
\textbf{Total Indexes:} 30+

\subsection{Appendix B: File Structure}

\begin{lstlisting}
Database-Programming-Research-Project/
├── database/
│   ├── schema.sql
│   ├── triggers.sql
│   ├── views.sql
│   ├── stored_procedures.sql
│   ├── events.sql
│   ├── security.sql
│   ├── seed_data.sql (DUMMY DATA)
│   ├── verify_data.sql
│   ├── role_based_analysis.sql
│   └── hiv_patient_care.sql (combined)
├── backend/ (API implementation)
├── docs/ (documentation)
└── eerd/ (EER diagrams)
\end{lstlisting}

\subsection{Appendix C: Project Repository}

\textbf{GitHub Repository:} \url{https://github.com/Cemputus/Database-Programming-Research-Project}

The complete source code, documentation, database scripts, and all project files are available in the GitHub repository. The repository includes:
\begin{itemize}
    \item Complete database schema and implementation files
    \item Backend API source code (Node.js/Express)
    \item Comprehensive documentation (README, setup guides, API documentation)
    \item EER diagram images and project documentation
    \item Database verification and analysis scripts
    \item Setup scripts and configuration files
\end{itemize}

\textbf{Repository Access:} The project is publicly available for academic review and can be cloned using:
\begin{lstlisting}
git clone https://github.com/Cemputus/Database-Programming-Research-Project.git
\end{lstlisting}

\subsection{Appendix D: References}

\begin{enumerate}
    \item Uganda Ministry of Health. (2023). \textit{HIV Treatment Guidelines}. Kampala, Uganda.
    \item Uganda AIDS Commission. (2023). \textit{National HIV/AIDS Strategic Plan}. Kampala, Uganda.
    \item United Nations. (2015). \textit{Sustainable Development Goals}. Goal 3: Good Health and Well-being.
    \item MySQL 8.0 Reference Manual. Oracle Corporation.
    \item Connolly, T. \& Begg, C. (2015). \textit{Database Systems: A Practical Approach to Design, Implementation, and Management}. 6th Edition.
\end{enumerate}

% ============================================================================
% END OF DOCUMENT
% ============================================================================

\end{document}

